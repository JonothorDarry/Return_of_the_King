\documentclass[12pt]{article}
\usepackage[utf8]{inputenc}
\usepackage[T1]{fontenc}
\usepackage{lmodern}
\usepackage[svgnames]{xcolor}
\usepackage[a4paper,bindingoffset=0.2in,%
            left=0.5in,right=0.5in,top=0.5in,bottom=1in,%
            footskip=.25in]{geometry}
\pagenumbering{gobble}
\usepackage[colorlinks=true, linkcolor=Black, urlcolor=Blue]{hyperref}
\usepackage{graphicx}
\usepackage{subcaption}
\usepackage{listings}
\usepackage{color}
\usepackage[svgnames]{xcolor}

\begin{document}
\title{Sprawozdanie z zadania na Przetwarzanie Równoległe\\
\large Projekt 1\\
\large Sebastian Michoń 136770, Marcin Zatorski 136834}
\date{\vspace{-10ex}}
\maketitle

\section{Wstęp}
\begin {enumerate}
\item Sebastian Michoń 136770: grupa dziekańska L1
\item Marcin Zatorski 136834: grupa dziekańska L10
\item Wymagany termin oddania sprawozdania: 27.04.2020r.
\item Rzeczywisty termin oddania sprawozdania: 27.04.2020r.
\item Wersja I sprawozdania
\item Adresy mailowe: sebastian.michon@student.put.poznan.pl, marcin.r.zatorski@student.put.poznan.pl
\end {enumerate}

\section{Wykorzystywany system równoległy}
\begin {enumerate}
	\item Kompilator: gcc 7.5.0
	\item System operacyjny: Ubuntu 18.04
	\item Procesor Intel(R) Core(TM) i7-4790K CPU @ 4.00GHz - 4 rdzenie, 2 wątki na 1 rdzeń: 8 procesorów logicznych i 4 fizyczne
	\item 
\end {enumerate}

\section{Tablica wynków: kody od Pi2 do Pi6 w 3 wersjach}
\begin{flushleft}
	\begin{tabular}{| l | l | l | l | l | l |}
		\hline
		Kod & Wątki & Rzeczywisty czas obliczeń & Czas użycia procesorów & Przysp. & Pi \\ \hline
		./pi\_s.c &  1 &  11.106908 &  11.106908 &  - &  3.141592653590 \\ \hline
		./pi2.c &  2 &  9.854029 &  19.660201 &  1.12714383 &  1.629332922363 \\ \hline
		./pi2.c &  4 &  6.201361 &  22.991912 &  1.79104361 &  0.606812628416 \\ \hline
		./pi2.c &  8 &  4.855977 &  37.927782 &  2.28726536 &  0.407415018512 \\ \hline
		./pi3.c &  2 &  32.231695 &  61.883320 &  .344595839 &  3.141592653590 \\ \hline
		./pi3.c &  4 &  68.859282 &  269.248873 &  .161298632 &  3.141592653590 \\ \hline
		./pi3.c &  8 &  102.148402 &  721.992069 &  .108733056 &  3.141592653590 \\ \hline
		./pi4.c &  2 &  5.644930 &  11.230873 &  1.96759003 &  3.141592653590 \\ \hline
		./pi4.c &  4 &  2.862752 &  11.442671 &  3.87980097 &  3.141592653590 \\ \hline
		./pi4.c &  8 &  1.489600 &  11.794959 &  7.45630236 &  3.141592653590 \\ \hline
		./pi5.c &  2 &  5.616025 &  11.189766 &  1.97771697 &  3.141592653590 \\ \hline
		./pi5.c &  4 &  2.865750 &  11.450181 &  3.87574212 &  3.141592653590 \\ \hline
		./pi5.c &  8 &  1.493857 &  11.804175 &  7.43505435 &  3.141592653590 \\ \hline
		./pi6.c &  2 &  6.310614 &  12.557454 &  1.76003602 &  3.141592653590 \\ \hline
		./pi6.c &  4 &  4.567762 &  15.864548 &  2.43158640 &  3.141592653590 \\ \hline
		./pi6.c &  8 &  3.467091 &  25.773811 &  3.20352364 &  3.141592653590 \\ \hline
	\end{tabular}
gdzie Czas użycia procesorów jest sumaryczny, a Przysp. to skrót od przyspieszenia kodu równoległego względem sekwencyjnego.
	
\end{flushleft}
\begin{enumerate}
	\item Kod pi2.c różni się od kodu sekwencyjnego dodaniem dyrektywy 
	\begin{lstlisting}[language=C]
		#pragma omp paralel for
	\end{lstlisting}
	dla pętli obliczającej pi. Zarówno $x$ jak i $sum$ są współdzielone przez wątki; błędny wynik wynika z wyścigu w dostępie do danych (który zachodzi np. jeśli jeden wątek zmieni wartość $x$-a w momencie w którym 2. wątek dodaję wartość zależną od $x$-a do sumy - jeśli 1. zapisze dane przed 2. wątkiem, ten 2. będzie korzystał z niepoprawnych danych), który zachodzi, ponieważ nie ma sychronizacji w dostępie do zmiennych współdzielonych. Sumaryczne czasy użycia procesorów są wyższe niż dla kodu sekwencyjnego, ponieważ nie zachodzi czasowa lokalność odwołań - dane muszą być zmieniane w pamięci po każdej zmianie sumy, ponadto jeśli $x$ został zmieniony pomiędzy zapisem $x$-a a zmianą sumy przez wątek, $x$ też zostanie zmieniony, choć nie powinien (jeśli algorytm ma działać poprawnie).
	\\
	
	
	\item Kod pi3.c ma lokalną dla wątku zmienną $x$ i zapis do sumy następuje z użyciem dyrektywy:
	\begin{lstlisting}[language=C]
		#pragma omp atomic
	\end{lstlisting}
		Co za tym idzie, zmienna współdzielona $sum$ jest uaktualniana w punkcie czasu przez dokładnie 1 wątek i nie zachodzi wyścig w dostępie do danych. Kod nie jest efektywny, ponieważ wykonuje $10^{9}$ razy operację założenia i zdjęcia zamka(blokady) na współdzielonej zmiennej $sum$ (przy czym najpewniej zachodzą jakieś optymalizacje, np. zamek jest zwalniany w momencie, gdy inny wątek chce zapisać do sumy, a nie bezpośrednio po jej aktualizacji - to by wyjaśniało dlaczego kod używający 2 wątków jest szybszy niż ten używający 8 wątków).\\
	
	\item Kod pi4.c różni się od pi3.c tym, że każdy wątek zapisuje i odczytuje lokalną dla wątku zmienną, po zakończeniu obliczeń zmienia atomowo sumę; nie zachodzi wyścig, ponieważ zmienne $x$ i $s2$ (suma dla pojedynczego wątku) są lokalne dla wątku, a zmienna $sum$ jest aktualizowana w punkcie czasu przez co najwyżej jeden wątek, ponieważ jest aktualizowana atomowo. Lokalne zmienne $x$ i $s2$ znajdują się w pamięci procesorów, które na nich operują i nie są zmieniane i odczytywane przez inne wątki, co implikuje czasową lokalność odwołań i efektywność: Sumaryczny czas użycia procesorów jest prawie taki sam jak kodu sekwencyjnego, użycie większej liczby procesorów prowadzi do przyspieszenia rzeczywistego wykonania kodu tyle razy, ile jest procesorów logicznych.\\
	
	\item Kod pi5.c jest podobny do kodu pi4.c, ale zamiast lokalnych zmiennych sumy jest tam:
	\begin{lstlisting}[language=C]
	#pragma omp parallel for private(x) reduction(+:sum)
	\end{lstlisting}
	co działa nieomal dokładnie w ten sam sposób co pi4.c, ponieważ dyrektywa tworzy zmienną lokalną dla wątku, która jest aktualizowana zamiast współdzielonej $sum$ w pętli i dodawana do zmiennej globalnej dla wątków dopiero po zakończeniu działania w pętli przez wątek. Co za tym idzie, czasowo ten algorytm funkcjonuje prawie tak samo jak pi4.c.\\
	
	\item Kod pi6.c różni się od pi4.c tym, że zamiast pojedynczych zmiennych lokalnych do sumowania została użyta tablica typu double[] taka, że każdy wątek aktualizuje tylko 1 indeks - co za tym idzie, zachodzi zjawisko false sharingu - zapisy do tablicy w 1 wątku powodują konieczność zmiany całej linii pamięci w innych pamięciach podręcznych procesorów, ponieważ te mają w swoich liniach pamięci własne kopie kolejnych wartości tej tablicy. Zmniejsza to  przestrzenną lokalność odwołań i efektywność kodu, sumaryczny czas użycia procesorów rośnie wraz ze zwiększeniem liczby wątków - a co za tym idzie, więcej razy zachodzi pobieranie linii do pamięci podręcznych procesora, ponieważ więcej wątków aktualizuje i odczytuje dane z tablicy.\\
	
\end{enumerate}
	 
\section{Długość linii}
Kod pi7.c wypisywał najniższe wartości czasu (niecałe 5.6 s) dla 2 wątków w iteracjach $k$ takich, że\\ 
\begin{center}
	$k\equiv7\ \textrm{mod}\ 8$
\end{center}
Przy czym w $k$-tej iteracji używałem wartości tablicy o indeksach $k$ i $k+1$, a iteracje indeksowałem od 0: co za tym idzie, długość linii pamięci podręcznej procesora, którego używałem to 64 bajty, ponieważ na moim systemie rozmiar chara to 1 bajt, a:
\begin{center}
	$sizeof(double)==8$
\end{center}
Czyli rozmiar typu double to 8 rozmiarów chara czyli 8 bajtów. Co za tym idzie, rozmiar linii pamięci w moim procesorze to $8*8=64$ bajty

\section{Wyjaśnienie, trudności}
\begin{enumerate}
	\item W 0. iteracji korzystam z indeksów (0, 1) tablicy $double[]$; w 1. z indeksów (1, 2), ... w 7. z indeksów (7, 8) i tak dalej.
	\item W 25 pierwszych iteracjach najniższe wyniki czasowe uzyskuje w 7., 15. i 23. iteracji. Można to logicznie uzasadnić, jeśli linia pamięci podręcznej procesora ma 8 bajtów: w 0., 1., ... 6. iteracji 2 procesory nadpisują wzajemnie własne linie pamięci podręcznej, bo pobierają za każdym razem kolejne 8 doubli z tablicy $double[]$ (0...7) - co za tym idzie, zmiana w tej tablicy wymusza zmianę w linii pamięci podręcznej 2. procesora. 
	\item W 7. iteracji 1. procesor ma w swojej linii pamięci elementy tablicy $double[]$ o indeksach 0...7, 2. procesor ma w swojej linii pamięci elementy tablicy $double[]$ o indeksach 8...15. 2 procesory nie nadpisują swoich linii pamięci, co przyspiesza działanie programu. Analogicznie, w 15. iteracji 1. procesor ma w swojej linii pamięci elementy tablicy $double[]$ o indeksach 8...15, 2. procesor - indeksy 16...23. To zjawisko powtarza się cyklicznie co 8 bajtów. Nie widzę innego sensownego uzasadnienia, dlaczego właśnie 7., 15. i 23. i kolejne co ósme iteracje dawały najlepsze rezultaty, dlatego przyjmuje powyższe rozumowanie za co najmniej zasadne, szczególnie zważając na to, że linia pamięci podręcznej mojego procesora ma 64 bajty zgodnie z jej dokumentacją.
	\item Gdy zamiast kolejnych wartości tablicy używałem wartości tablicy o indeksach, których różnica wynosi 20 dla 2 wątków, wyniki czasowe iteracji były takie same jak dla co 8. iteracji używając kolejnych elementów tablicy, co wzmacnia tezę o tym, że przestrzenna lokalność odwołań ma wpływ na długość obliczeń i to, że linia pamięci faktycznie ma 8 bajtów.
	\item Jedyną trudnością były zdarzające się iteracje, w których procesor delikatnie zwalniał, zapewne w wyniku procesów w tle, po wyeliminowaniu których otrzymałem powyższe rezultaty. Różnice pomiędzy 7., 15., 23. ... a pozostałymi iteracjami były rzędu 0.15-0.4 sekundy, dlatego zasadnym było albo wydłużenie pętli, albo wyeliminowanie zbędnych procesów w tle.
\end{enumerate}


\end{document}
