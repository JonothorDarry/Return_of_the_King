\documentclass[12pt]{article}
\usepackage[utf8]{inputenc}
\usepackage[T1]{fontenc}
\usepackage{lmodern}
\usepackage[svgnames]{xcolor}
\usepackage[a4paper,bindingoffset=0.2in,%
            left=0.5in,right=0.5in,top=0.5in,bottom=1in,%
            footskip=.25in]{geometry}
\pagenumbering{gobble}
\usepackage[colorlinks=true, linkcolor=Black, urlcolor=Blue]{hyperref}

\begin{document}
\title{Sprawozdanie z zadania na Przetwarzanie Równoległe\\
\large Zadanie 2\\
\large Sebastian Michoń 136770}
\date{\vspace{-10ex}}
\maketitle

\section{Wykorzystywany system równoległy}
\begin {enumerate}
	\item Kompilator: gcc 7.5.0
	\item System operacyjny: Ubuntu 18.04
	\item Procesor Intel(R) Core(TM) i7-4790K CPU @ 4.00GHz - 4 rdzenie, 2 wątki na 1 rdzeń: 8 procesorów logicznych i 4 fizyczne
\end {enumerate}

\section{Tablica wynków: kody od Pi2 do Pi6 w 3 wersjach}
\begin{center}
	\begin{tabular}{| l | l | l | l |}
		\hline
		Wersja kodu & Czas trwania obliczeń & Czas użycia procesorów & Przyspieszenie \\ \hline
		Monday & 11C & 22C & 5 \\ \hline
		Tuesday & 9C & 19C & 4 \\ \hline
		Wednesday & 10C & 21C & 3 \\
		\hline
	\end{tabular}
\end{center}
	 
\section{Struktura klienta}
\begin{enumerate}
	\item 
\end{enumerate}

\section{Struktura node'a}
\begin{enumerate}
	\item 
\end{enumerate}

\section{Uruchomienie}
\begin{enumerate}
	\item Uruchomienie serwera:\\
		cd ...(miejsce, gdzie zaszyty jest projekt)\\
		
		
\end{enumerate}

\end{document}
