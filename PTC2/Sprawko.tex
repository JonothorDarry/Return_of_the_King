\documentclass[12pt]{article}
\usepackage[utf8]{inputenc}
\usepackage[T1]{fontenc}
\usepackage{lmodern}
\usepackage[svgnames]{xcolor}
\usepackage[a4paper,bindingoffset=0.2in,%
            left=0.5in,right=0.5in,top=0.5in,bottom=1in,%
            footskip=.25in]{geometry}
\pagenumbering{gobble}
\usepackage[colorlinks=true, linkcolor=Black, urlcolor=Blue]{hyperref}

\begin{document}
\title{Sprawozdanie z zadania na Przetwarzanie Równoległe\\
\large Zadanie 2\\
\large Sebastian Michoń 136770}
\date{\vspace{-10ex}}
\maketitle

\section{Wykorzystywany system równoległy}
\begin {enumerate}
	\item Kompilator: gcc 7.5.0
	\item System operacyjny: Ubuntu 18.04
	\item Procesor Intel(R) Core(TM) i7-4790K CPU @ 4.00GHz - 4 rdzenie, 2 wątki na 1 rdzeń: 8 procesorów logicznych i 4 fizyczne
\end {enumerate}

\section{Tablica wynków: kody od Pi2 do Pi6 w 3 wersjach}
\begin{flushleft}
	\begin{tabular}{| l | l | l | l | l | l |}
		\hline
		Kod & Wątki & Rzeczywisty czas obliczeń & Czas użycia procesorów & Przysp. & Pi \\ \hline
		./pi\_s.c &  1 &  11.106908 &  11.106908 &  - &  3.141592653590 \\ \hline
		./pi2.c &  2 &  9.854029 &  19.660201 &  1.12714383 &  1.629332922363 \\ \hline
		./pi2.c &  4 &  6.201361 &  22.991912 &  1.79104361 &  0.606812628416 \\ \hline
		./pi2.c &  8 &  4.855977 &  37.927782 &  2.28726536 &  0.407415018512 \\ \hline
		./pi3.c &  2 &  32.231695 &  61.883320 &  .344595839 &  3.141592653590 \\ \hline
		./pi3.c &  4 &  68.859282 &  269.248873 &  .161298632 &  3.141592653590 \\ \hline
		./pi3.c &  8 &  102.148402 &  721.992069 &  .108733056 &  3.141592653590 \\ \hline
		./pi4.c &  2 &  5.644930 &  11.230873 &  1.96759003 &  3.141592653590 \\ \hline
		./pi4.c &  4 &  2.862752 &  11.442671 &  3.87980097 &  3.141592653590 \\ \hline
		./pi4.c &  8 &  1.489600 &  11.794959 &  7.45630236 &  3.141592653590 \\ \hline
		./pi5.c &  2 &  5.616025 &  11.189766 &  1.97771697 &  3.141592653590 \\ \hline
		./pi5.c &  4 &  2.865750 &  11.450181 &  3.87574212 &  3.141592653590 \\ \hline
		./pi5.c &  8 &  1.493857 &  11.804175 &  7.43505435 &  3.141592653590 \\ \hline
		./pi6.c &  2 &  6.310614 &  12.557454 &  1.76003602 &  3.141592653590 \\ \hline
		./pi6.c &  4 &  4.567762 &  15.864548 &  2.43158640 &  3.141592653590 \\ \hline
		./pi6.c &  8 &  3.467091 &  25.773811 &  3.20352364 &  3.141592653590 \\ \hline
		
	\end{tabular}
	
\end{flushleft}
	 
\section{Struktura klienta}
\begin{enumerate}
	\item 
\end{enumerate}

\section{Struktura node'a}
\begin{enumerate}
	\item 
\end{enumerate}

\section{Uruchomienie}
\begin{enumerate}
	\item Uruchomienie serwera:\\
		cd ...(miejsce, gdzie zaszyty jest projekt)\\
		
		
\end{enumerate}

\end{document}
